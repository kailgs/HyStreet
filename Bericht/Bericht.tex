\documentclass{article}

% Präambel, um Jupyter-Blocks zu importieren
\input{jupyter-preamble.tex}

% Packages
\usepackage{amsmath}
\usepackage{graphicx}
\usepackage{anysize}
\usepackage{setspace}
\usepackage{xcolor}
\usepackage{sectsty}
\usepackage{titlesec}
% \usepackage{hyperref}

% Einstellungen
\definecolor{hyblue}{RGB}{24, 62, 98}
\sectionfont{\color{hyblue}} % Überschriftfarbe
\marginsize{25mm}{25mm}{15mm}{25mm} % Seitenränder (links, rechts, oben, unten)
\titlespacing*{\section}{0pt}{0.5cm}{0.8cm} % Platz nach Section
\titlespacing*{\subsection}{0pt}{1.3cm}{0.5cm} % Platz nach Subsection
\renewcommand{\contentsname}{Inhaltsverzeichnis} % Titel des Inhaltsverzeichnisses ändern
\hypersetup{
    colorlinks,
    citecolor=black,
    linkcolor=black,
    filecolor=black,
    urlcolor=black
}

% Titelseite
\title{{\Large Datenpipeline zur Verarbeitung von Daten der Datenquelle HyStreet.com}} % Titel
\date{01-01-2023}
\author{Mathieu, Daniel, Kai}

% Dokument
\begin{document}

% Titelseite
\pagenumbering{gobble}    
\maketitle
\begin{figure}[h!]
    \centering
    \includegraphics[width=0.7\linewidth]{images/hystreet-Logo.png}
\end{figure}
\newpage

% Inhaltsverzeichnis
\pagenumbering{roman}
\doublespacing
\tableofcontents
\singlespacing
\newpage

%%%%%%%%%%%%%%%%%%%%%%%%%%%%%%%%% Inhalt %%%%%%%%%%%%%%%%%%%%%%%%%%%%%%%%%
\onehalfspacing
\pagenumbering{arabic}
\section{Einführung}
Ziel dieser Arbeit ist die Erstellung einer Datenpipeline, die die Daten der Datenquelle \href{https://hystreet.com}{HyStreet} verarbeitet, speichert und anschließend analysiert und visualisiert.

\section{Datenpipeline}

Im nachfolgenden Teil der Arbeit werden die einzelnen Elemente der erstellte Datenpipeline nun genauer besprochen und erläutert. Dazu wird in Abbildung \ref{fig:datenpipelineFull} zunächst die Datenpipeline als Ganzen betrachtet.

\begin{figure}[h!]
    \vspace{1cm}
    \centering
    \includegraphics[width=0.9\linewidth]{images/datenpipeline.png}
    \caption{Datenpipeline zur Verarbeitung der HyStreet-Daten}
    \label{fig:datenpipelineFull}
    \vspace{0.5cm}
\end{figure}

Wie bereits erwähnt, stammen die Daten aus der Internetseite \href{https://hystreet.com}{HyStreet.com}. Die Webseite stellt somit auch den Startpunkt der Datenpipeline dar. Da die Daten nur kostenpflichtig automatisiert über eine API Schnittstelle heruntergeladen werden konnten, mussten wir uns für die Alternative entscheiden, die Daten manuell über deren Webseite herunterzuladen. Über den manuellen Download ist der Erhalt der Daten kostenlos. Zum genauen Vorgang des Downloads im folgenden Abschnitt Genaueres. Das Ergebnis dieses Schrittes ist eine Dateien im CSV-Format für den jeweiligen Standort. Der Aufbau dieser Dateien wurde bereits im einführenden Kapitel genauer beschrieben. Nach dem Herunterladen, werden die Dateien nun zunächst im lokalen Dateisystem gespeichert, um anschließend im Python Skript \emph{Skript 1} verarbeitet werden zu können.
\bigbreak

In diesem Skript werden die Daten aus allen heruntergeladenen Dateien mit Hilfe des Packages \emph{Pandas} eingelesen und verarbeitet. Dabei werden diverse Spalten in ihre Einzelteile zerlegt und Spalten, die für unsere Zwecke nicht weiter erforderlich sind entfernt. Nach der Aufbereitung der Daten, werden diese schließlich in das JSON-Format umgewandelt, um im letzten Schritt in eine MongoDB eingelesen und gespeichert zu werden.
\bigbreak

Nachdem die Daten nun in der MongoDB persistiert wurden, können diese nun im \emph{Skript 2} analysiert und visualisiert werden. Sowohl die Analyse als auch die Visualisierung sind dabei in einem Jupyter Notebook integriert.
\include{sections/speicherung.tex}
\section{Analyse und Visualisierung}
Betrachten wir nun \emph{Skript 2}, in dem die aufbereiteten Daten innerhalb eines Jupyter Notebooks analysiert und visualisiert werden. Im Notebook werden zunächst die Daten aus der MongoDB eingelesen, einige Hilfsfunktionen definiert und die Standorte auf einer Karte mit Hilfe des Packages \emph{Folium} angezeigt.

\subsection{Vergleich Passantenanzahl pro Jahr}
Um einen groben Überblick über die Daten zu erhalten, werden zunächst die Passantenanzahl pro Jahr für die einzelnen Standorte berechnet und die Standorte sowie Jahre miteinander verglichen. Für die Berechnung der Summe pro Jahr wird folgende Funktion definiert, die die gewünschten Daten in einem Pandas-Dataframe zurückgibt:

\bigbreak
\begin{tcolorbox}[breakable, size=fbox, boxrule=1pt, pad at break*=1mm,colback=cellbackground, colframe=cellborder]
\prompt{In}{incolor}{1}{\boxspacing}
\begin{Verbatim}[commandchars=\\\{\}]
\PY{k}{def} \PY{n+nf}{sumOfYear}\PY{p}{(}\PY{n}{data}\PY{p}{,} \PY{n}{year}\PY{p}{)}\PY{p}{:}
  \PY{n}{result} \PY{o}{=} \PY{p}{[}\PY{p}{]}
  \PY{k}{for} \PY{n}{address} \PY{o+ow}{in} \PY{n}{addresses}\PY{p}{:}
    \PY{n}{filteredByYear} \PY{o}{=} \PY{n}{data}\PY{o}{.}\PY{n}{loc}\PY{p}{[}\PY{n}{data}\PY{p}{[}\PY{l+s+s1}{\PYZsq{}}\PY{l+s+s1}{address}\PY{l+s+s1}{\PYZsq{}}\PY{p}{]} \PY{o}{==} \PY{n}{address}\PY{p}{]}\PY{o}{.}\PY{n}{loc}\PY{p}{[}\PY{n}{data}\PY{p}{[}\PY{l+s+s1}{\PYZsq{}}\PY{l+s+s1}{year}\PY{l+s+s1}{\PYZsq{}}\PY{p}{]} \PY{o}{==} \PY{n}{year}\PY{p}{]}
    \PY{n}{result}\PY{o}{.}\PY{n}{append}\PY{p}{(}\PY{p}{(}\PY{n}{address}\PY{p}{,} \PY{n}{filteredByYear}\PY{p}{[}\PY{l+s+s2}{\PYZdq{}}\PY{l+s+s2}{pedestrians}\PY{l+s+s2}{\PYZdq{}}\PY{p}{]}\PY{o}{.}\PY{n}{sum}\PY{p}{(}\PY{p}{)}\PY{p}{)}\PY{p}{)}
  \PY{k}{return} \PY{n}{result}
\end{Verbatim}
\end{tcolorbox}
\bigbreak

Damit kann nun die absolute Passantenanzahl pro Jahr für jeden Standort ermittelt werden. Fügt man die Dataframes für die einzelnen Jahre nun zusammen, können die Standorte und Jahre mit dem Package \emph{Plotly} und dessen Histogram-Plot einfach miteinander verglichen werden und liefern uns folgendes Ergebnis:

\begin{figure}[h!]
    \vspace{0.2cm}
    \centering
    \includegraphics[width=\linewidth]{images/absJahresvergleich.png}
    \caption{Vergleich der Standorte für die einzelnen Jahre}
    \label{fig:absJahresvergleich}
\end{figure}

\subsection{Uhrzeitenvergleich}
Vor allem für Ladenbesitzer ist es nicht nur wichtig zu wissen, an welchem Wochentag die meisten Menschen an den jeweiligen Standorten unterwegs sind, sondern auch um welche Uhrzeit. 
Betrachtet werden hier alle Standorte gleichzeitig, um ein allgemeines Gefühl zu erhalten, welche Uhrzeiten besonders beliebt sind. Möchte man nur einen speziellen Standort betrachten, muss man die Eingabedaten zuvor nur auf den jeweiligen Standort filtern und die Funktion erzielt die gleichen Ergebnisse. In SQL übersetzt würde die Funktion folgendens berechnen, $ SUM(pedestrians) GROUP BY Month, Hour $, betrachtet also für alle Standorte gleichzeitig wie viele Passanten pro Monat pro jeweiliger Stunde gemessen wurde:

\bigbreak
\begin{tcolorbox}[breakable, size=fbox, boxrule=1pt, pad at break*=1mm,colback=cellbackground, colframe=cellborder]
\prompt{In}{incolor}{2}{\boxspacing}
\begin{Verbatim}[commandchars=\\\{\}]
\PY{k}{def} \PY{n+nf}{absPerHour}\PY{p}{(}\PY{n}{data}\PY{p}{,} \PY{n}{years}\PY{p}{)}\PY{p}{:}
  \PY{n}{dataTime} \PY{o}{=} \PY{n}{data}\PY{o}{.}\PY{n}{loc}\PY{p}{[}\PY{n}{data}\PY{p}{[}\PY{l+s+s1}{\PYZsq{}}\PY{l+s+s1}{year}\PY{l+s+s1}{\PYZsq{}}\PY{p}{]}\PY{o}{.}\PY{n}{isin}\PY{p}{(}\PY{n}{years}\PY{p}{)}\PY{p}{]}
  \PY{n}{dataTL} \PY{o}{=} \PY{n+nb}{list}\PY{p}{(}\PY{p}{)}
  \PY{n}{monate} \PY{o}{=} \PY{p}{[}\PY{l+s+s2}{\PYZdq{}}\PY{l+s+s2}{Januar}\PY{l+s+s2}{\PYZdq{}}\PY{p}{,} \PY{l+s+s2}{\PYZdq{}}\PY{l+s+s2}{Februar}\PY{l+s+s2}{\PYZdq{}}\PY{p}{,} \PY{l+s+s2}{\PYZdq{}}\PY{l+s+s2}{März}\PY{l+s+s2}{\PYZdq{}}\PY{p}{,} \PY{l+s+s2}{\PYZdq{}}\PY{l+s+s2}{April}\PY{l+s+s2}{\PYZdq{}}\PY{p}{,} \PY{l+s+s2}{\PYZdq{}}\PY{l+s+s2}{...}\PY{l+s+s2}{"}{]}
  \PY{n}{df} \PY{o}{=} \PY{n}{pd}\PY{o}{.}\PY{n}{DataFrame}\PY{p}{(}\PY{n}{columns}\PY{o}{=}\PY{p}{[}\PY{l+s+s2}{\PYZdq{}}\PY{l+s+s2}{Monat}\PY{l+s+s2}{\PYZdq{}}\PY{p}{,} \PY{l+s+s2}{\PYZdq{}}\PY{l+s+s2}{Stunde}\PY{l+s+s2}{\PYZdq{}}\PY{p}{,} \PY{l+s+s2}{\PYZdq{}}\PY{l+s+s2}{Passanten}\PY{l+s+s2}{\PYZdq{}}\PY{p}{]}\PY{p}{)}
  \PY{k}{for} \PY{n}{month} \PY{o+ow}{in} \PY{n+nb}{range}\PY{p}{(}\PY{l+m+mi}{12}\PY{p}{)}\PY{p}{:}
    \PY{n}{tmp} \PY{o}{=} \PY{n}{dataTime}\PY{o}{.}\PY{n}{loc}\PY{p}{[}\PY{n}{dataTime}\PY{p}{[}\PY{l+s+s1}{\PYZsq{}}\PY{l+s+s1}{month}\PY{l+s+s1}{\PYZsq{}}\PY{p}{]} \PY{o}{==} \PY{p}{(}\PY{n}{month} \PY{o}{+} \PY{l+m+mi}{1}\PY{p}{)}\PY{p}{]}
    \PY{n}{tsum} \PY{o}{=} \PY{n}{tmp}\PY{o}{.}\PY{n}{groupby}\PY{p}{(}\PY{l+s+s1}{\PYZsq{}}\PY{l+s+s1}{hour}\PY{l+s+s1}{\PYZsq{}}\PY{p}{)}\PY{p}{[}\PY{l+s+s1}{\PYZsq{}}\PY{l+s+s1}{pedestrians}\PY{l+s+s1}{\PYZsq{}}\PY{p}{]}\PY{o}{.}\PY{n}{sum}\PY{p}{(}\PY{p}{)}
    \PY{k}{for} \PY{n}{hour} \PY{o+ow}{in} \PY{n+nb}{range}\PY{p}{(}\PY{l+m+mi}{0}\PY{p}{,} \PY{l+m+mi}{24}\PY{p}{)}\PY{p}{:}
      \PY{n}{df} \PY{o}{=} \PY{n}{df}\PY{o}{.}\PY{n}{append}\PY{p}{(}\PY{p}{\PYZob{}}\PY{l+s+s2}{\PYZdq{}}\PY{l+s+s2}{Monat}\PY{l+s+s2}{\PYZdq{}} \PY{p}{:} \PY{n}{monate}\PY{p}{[}\PY{n}{month}\PY{p}{]}\PY{p}{,} \PY{l+s+s2}{\PYZdq{}}\PY{l+s+s2}{Stunde}\PY{l+s+s2}{\PYZdq{}} \PY{p}{:} \PY{n}{hour}\PY{p}{,} \PY{l+s+s2}{\PYZdq{}}\PY{l+s+s2}{Passanten}\PY{l+s+s2}{\PYZdq{}} \PY{p}{:} \PY{n}{tsum}\PY{o}{.}\PY{n}{iloc}\PY{p}{[}\PY{n}{hour}\PY{p}{]} \PY{p}{\PYZcb{}}\PY{p}{,} \PY{n}{ignore\PYZus{}index}\PY{o}{=}\PY{k+kc}{True}\PY{p}{)}
  \PY{k}{return} \PY{n}{df}
\PY{n}{absHours} \PY{o}{=} \PY{n}{absPerHour}\PY{p}{(}\PY{n}{data}\PY{p}{,} \PY{p}{[}\PY{l+m+mi}{2020}\PY{p}{]}\PY{p}{)}
\end{Verbatim}
\end{tcolorbox}
\bigbreak

Das resultierende Pandas-Dataframe kann anschließend wieder mit Plotly und dessen Linien-Plot visualisiert werden. Zu erkennen ist, dass der Höhepunkt, unabhängig des Monats gegen ca. 16 Uhr erreicht wird. 

\begin{figure}[h!]
    \vspace{0.2cm}
    \centering
    \includegraphics[width=\linewidth]{images/absHour.png}
    \caption{Vergleich der Passantenanzahl pro Monat pro Stunde}
    \label{fig:absStundenvergleich}
\end{figure}


% Bildverzeichnis, Tabellenverzeichnis
\begin{appendix}
    \listoffigures
    \listoftables
\end{appendix}

% Dokument beenden
\end{document}